\documentclass[10pt,letterpaper]{article}

% ATS Compatibility:
\input{glyphtounicode}
\pdfgentounicode=1

\usepackage[letterpaper,margin=0.5in]{geometry}
\usepackage[utf8]{inputenc}
\usepackage{mdwlist}
\usepackage[T1]{fontenc}
\usepackage[usenames, dvipsnames]{color}
\usepackage{textcomp}
\usepackage{xcolor}
\usepackage{sectsty}
\usepackage[hidelinks]{hyperref}
\hypersetup{
	colorlinks = true,
	urlcolor={black},
}
\usepackage{tgpagella}
\pagestyle{empty}
\setlength{\tabcolsep}{0em}
\renewcommand{\baselinestretch}{1.025}

\subsectionfont{\fontsize{11}{15}\selectfont}


\newcommand{\items}[2]
{
	% @ is a column separator 
	\begin{tabular*}{\linewidth}{l @{\extracolsep{\fill}} r}
		#1 & #2 \\
	\end{tabular*}
}


\newcommand{\header}[2]
{
	% @ is a column separator 
	\begin{tabular*}{\linewidth}{l @{\extracolsep{\fill}} r}
		\hspace{-27pt} #1 & #2 \\
	\end{tabular*}
}

\newcommand{\sectionbreak}
{
	\vspace{-1.2em}
	\rule{\textwidth}{1.7pt}
	\vspace{-1.7em}
}

\newcommand{\twocol}[2]
{
	\begin{tabular*}{\linewidth}{l @{\hspace{108.5pt}} l}
		 #1 & #2 \\
	\end{tabular*}
	\vspace{-15pt}

}

\begin{document}

\begin{center}
{\LARGE \textbf{Aashir Farooqi}}

\vspace{0.5em}
\ (949)-226-9612 \textbar 
\ afarooqi@ucdavis.edu\textbar
\ \href{https://github.com/AashPointO}{\emph{\underline{\smash{https://github.com/AashPointO}}}}
\\
\end{center}
\vspace{-20pt}


\subsection*{Education}
\sectionbreak

\begin{itemize}

\item[] 
	\header
		{\textbf{University of California, Davis}}
		{\textbf{Fall 2016 - Summer 2020}}
\item[]
	\vspace{-2.5pt}
	\items
		{\textbf{Major:} Computer Engineering, B.S}
		{}
	\items
		{\textbf{GPA:} 3.4}
		{}
	\items
		{\textbf{CS Coursework:} Algorithm Design \& Analysis, Applied Linear Algebra, Operating Systems, Networks.}
		{}
	\items
		{\textbf{EE Coursework:} Embedded Systems, Digital Systems, Circuits, Signal Processing.}
		{}
{\vspace{-0.6em}}
	
\end{itemize}

\vspace{-24.65pt}

\subsection*{Experience}
\sectionbreak

\begin{itemize}
	\item[]
		\header
			{\textbf{Firmware \& Hardware Engineer - Research Assistant}} 
			{\textbf{April 2018 - June 2020}}
		\header
		{\textbf{Miller Lab} \ (\href{https://millerlab.faculty.ucdavis.edu}{\small \emph{\underline{\smash{millerlab.faculty.ucdavis.edu})}}} }
			{\textbf{Auditory Neuroscience \& Speech Recognition Lab}} 
		\item
			Independently brought up, prototyped, and developed a real-time solution to cross-reference external audio inputs with an EEG acquisition system by writing embedded firmware code in C and designing/assembling a high speed analog circuit board, with simulations done in OrCad alongside a custom PCB. Brought latency down from the previous iteration by a factor of 10.
		\item 
			Configured, built, and ran a custom RTOS (embedded linux distribution) and cross-toolchain using the Yocto Project, by understanding both the architectural design of the embedded system, alongside the computer architecture of the host system.
		\item
			Utilized the version control software of GIT in all software projects, and leveraged software/hardware debugging skills, such as GDB and typical lab bench equipment (oscilloscopes, logic analyzers, function generator).
\end{itemize}

\hrule

\begin{itemize}
	\item[]
		\header
			{\textbf{Software Engineer - Intern}} 
			{\textbf{June 2018 - August 2018}}
		\header
			{\textbf{General Atomics}}
			{\textbf{EMS - Software and Controls}} 
		\item
			Leveraged object-oriented and algorithm design principles in an agile software development team environment, to convert the code base for an aircraft landing from MATLAB to C++, bringing the runtime of the simulation down by a factor of 2. Despite tight time constraints and minimal assistance, I earned the "MVP" award for saving "hundreds of hours in simulation time and greatly reducing control system tuning efforts".


\end{itemize}

\vspace{-1.5em}

\subsection*{Projects}
\sectionbreak


\begin{itemize}
	\item[]
		\header
		{
			\textbf{Smart Dog Collar}
			\emph{\smash{Relevant Course Project}} \ \ \ \footnotesize \emph{C \& Verilog}
		}
			{\textbf{Fall 2019 \& Winter 2020}}
		\item 
			Wrote embedded firmware, alongside designing an FPGA to interact with external peripherals using SPI, I2C, and UART, in order to track ambient noise, and cycle between different power modes for the SoC.
		\item 
			Designed multiple iterations of PCB and schematic captures in Altium, to create high speed digital circuit boards.

\end{itemize}

\hrule

\begin{itemize}
	\item[]
		\header
		{
			\textbf{Operating Systems}
			\emph{\smash{Relevant Course Project}} \ \ \ \footnotesize \emph{C++}
		}
			{\textbf{Spring 2020}}
		\item 
			Wrote C++ code to implement the functionality of a Linux based Operating System, including the development and implementation of preemptive threading, alongside writing a Linux device driver for a FAT16 file system.

\end{itemize}
\hrule
\begin{itemize}
	\item[]
		\header
		{
			\textbf{Digital Logic}
			\emph{\smash{Relevant Course Project/Material}} \ \ \ \footnotesize \emph{C++}
		}
			{\textbf{Spring 2020}}
		\item 
			Wrote Verilog onto an FPGA to manually drive a VGA display, interact with an accelerometer, and create a custom user interface.

\end{itemize}
\hrule

\begin{itemize}
	\item[]
		\header
		{
			\textbf{Mobile Applications (IOS): }
			\href{https://appadvice.com/app/round-bound/1369632746}{\emph{\underline{\smash{Round 'a Bound}}}}, 
			\href{https://appadvice.com/app/tic-tac-emoji/1346934986}{\emph{\underline{\smash{Tic-Tac Emoji}}}} \ \ \ \footnotesize  \emph{Swift}
		}
			{\textbf{Winter 2017 \& Spring 2018}}
		\item 
			Successfully delivered the products through their full life cycle, by utilizing different APIs to detect physics collisions, exhibiting independently made animations and sound, and incorporating an online leaderboard via a real time database. While formerly published, the apps culminated in over 250 downloads.
\end{itemize}

\hrule

\begin{itemize}
	\item[]
		\header
			{	
				\textbf{Website: }
				\href{https://aashpointo.github.io/KmapWebsite/}{\emph{\underline{\smash{aashpointo.github.io/KmapWebsite}}}} \ \ \ \footnotesize \emph{HTML/CSS \& JavaScript}
				}
				{\textbf{Winter 2018}}
		\item
			Delivered the web application through its full lifecycle, by implementing the Quine-McCluskey algorithm in JavaScript to compute the \emph{Sum of Products} and \emph{Product of Sums} from a set of truth-table inputs.

\end{itemize}

\end{document}

