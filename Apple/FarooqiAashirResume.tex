\documentclass[10pt,letterpaper]{article}

% ATS Compatibility:
\input{glyphtounicode}
\pdfgentounicode=1

\usepackage[letterpaper,margin=0.5in]{geometry}
\usepackage[utf8]{inputenc}
\usepackage{mdwlist}
\usepackage[T1]{fontenc}
\usepackage[usenames, dvipsnames]{color}
\usepackage{textcomp}
\usepackage{xcolor}
\usepackage{sectsty}
\usepackage[hidelinks]{hyperref}
\hypersetup{
	colorlinks = true,
	urlcolor={black},
}
\usepackage{tgpagella}
\pagestyle{empty}
\setlength{\tabcolsep}{0em}
\renewcommand{\baselinestretch}{1.025}

\subsectionfont{\fontsize{11}{15}\selectfont}


\newcommand{\items}[2]
{
	% @ is a column separator 
	\begin{tabular*}{\linewidth}{l @{\extracolsep{\fill}} r}
		#1 & #2 \\
	\end{tabular*}
}


\newcommand{\header}[2]
{
	% @ is a column separator 
	\begin{tabular*}{\linewidth}{l @{\extracolsep{\fill}} r}
		\hspace{-27pt} #1 & #2 \\
	\end{tabular*}
}

\newcommand{\sectionbreak}
{
	\vspace{-1.2em}
	\rule{\textwidth}{1.7pt}
	\vspace{-1.7em}
}

\newcommand{\twocol}[2]
{
	\begin{tabular*}{\linewidth}{l @{\hspace{108.5pt}} l}
		 #1 & #2 \\
	\end{tabular*}
	\vspace{-15pt}

}

\begin{document}

\begin{center}
{\LARGE \textbf{Aashir Farooqi}}

\vspace{0.5em}
\ (949)-226-9612 \textbar 
\ afarooqi@ucdavis.edu\textbar
\ \href{https://github.com/AashPointO}{\emph{\underline{\smash{https://github.com/AashPointO}}}}
\\
\end{center}
\vspace{-20pt}


\subsection*{Education}
\sectionbreak

\begin{itemize}

\item[] 
	\header
		{\textbf{University of California, Davis}}
		{\textbf{Fall 2016 - Summer 2020}}
\item[]
	\vspace{-2.5pt}
	\items
		{\textbf{Major:} Computer Engineering, B.S}
		{}
	\items
		{\textbf{GPA:} 3.4}
		{}
	\items
		{\textbf{CS Coursework:} Algorithm Design \& Analysis, Applied Linear Algebra, Operating Systems, Networks.}
		{}
	\items
		{\textbf{EE Coursework:} Embedded Systems, Digital Systems, Circuits, Signal Processing.}
		{}
{\vspace{-0.6em}}
	
\end{itemize}

\vspace{-24.65pt}

\subsection*{Experience}
\sectionbreak

\begin{itemize}
	\item[]
		\header
			{\textbf{Firmware \& Hardware Engineer - Research Assistant}} 
			{\textbf{April 2018 - June 2020}}
		\header
		{\textbf{Miller Lab} \ (\href{https://millerlab.faculty.ucdavis.edu}{\small \emph{\underline{\smash{millerlab.faculty.ucdavis.edu})}}} }
			{\textbf{Auditory Neuroscience \& Speech Recognition Lab}} 
		\item
			Independently brought up, prototyped, and developed a real-time solution to cross-reference external audio inputs with an EEG acquisition system by writing C code on a bare-metal embedded system and designing/assembling a single-bit ADC PCB in Altium. Brought latency down from the previous iteration by a factor of 10.
		\item 
			Configured, built, and designed a custom RTOS (embedded Linux distribution) and cross-toolchain using the Yocto Project in a Linux environment, by using the Bash scripting language, and by understanding the hardware specs of the embedded system, alongside the hardware specs of the host system.
		\item
			Designed/read a number of schematics, consulted a number of datasheets for the various IC components used, utilized GDB for software-level debugging, and used hardware-level debugging tools such as scopes, logic analyzers, and function generators to debug circuits that interface with the embedded software.
		\item 
			Implemented an external eye-tracking peripheral using TCP/IP communication and a Python to MATLAB library for use in behavioral studies.
\end{itemize}

\hrule

\begin{itemize}
	\item[]
		\header
			{\textbf{Software Engineer - Intern}} 
			{\textbf{June 2018 - August 2018}}
		\header
			{\textbf{General Atomics}}
			{\textbf{EMS - Software and Controls}} 
		\item
			Leveraged object-oriented and algorithm design principles in an agile software development team environment, to convert the code base for an aircraft landing from the scripting language of MATLAB to C++, bringing the runtime of the simulation down by a factor of 2. Despite a tight schedule and minimal assistance, I earned the "MVP" award for saving "hundreds of hours in simulation time and greatly reducing control system tuning efforts".
		\item
			Wrote unit-tests in C++ to verify the accuracy of the simulations, using the Catch testing-framework.


\end{itemize}

\vspace{-1.5em}

\subsection*{Projects}
\sectionbreak


\begin{itemize}
	\item[]
		\header
		{
			\textbf{Smart Dog Collar}
			\emph{\smash{Relevant Course Project}} \ \ \ \footnotesize \emph{C \& Verilog}
		}
			{\textbf{Fall 2019 \& Winter 2020}}
		\item 
			Wrote C code on a bare-metal embedded system, with peripheral interfacing using standard hardware protocols, such as SPI, I2C, I2S, and UART, to track ambient noise, and cycle between different power modes for the SoC and peripherals.
		\item 
			Worked collaboratively with Electrical Engineers to design/read hardware schematics, consult datasheets, design/assemble multiple PCBs, and debug circuits using logic analyzers, scopes, and other hardware-level debugging tools.

\end{itemize}

\hrule

\begin{itemize}
	\item[]
		\header
		{
			\textbf{Operating Systems}
			\emph{\smash{Relevant Course Project}} \ \ \ \footnotesize \emph{C++}
		}
			{\textbf{Spring 2020}}
		\item 
			Wrote C++ code to implement the functionality of a Unix based Operating System, including the implementation of preemptive threading, developing a device driver for a FAT16 file system, and writing typical Unix shell command line tools.

\end{itemize}
\hrule
\begin{itemize}
	\item[]
		\header
		{
			\textbf{Mobile Applications (IOS): }
			\href{https://appadvice.com/app/round-bound/1369632746}{\emph{\underline{\smash{Round 'a Bound}}}}, 
			\href{https://appadvice.com/app/tic-tac-emoji/1346934986}{\emph{\underline{\smash{Tic-Tac Emoji}}}} \ \ \ \footnotesize  \emph{Swift}
		}
			{\textbf{Winter 2017 \& Spring 2018}}
		\item 
			Successfully delivered the products through their full life cycle, by utilizing different APIs to detect physics collisions, exhibiting independently made animations and sound, and incorporating an online leaderboard via a real time database, all in an Xcode environment. While formerly published, the apps culminated in over 250 downloads.
\end{itemize}

\hrule

\begin{itemize}
	\item[]
		\header
			{	
				\textbf{Web Application: }
				\href{https://aashpointo.github.io/KmapWebsite/}{\emph{\underline{\smash{aashpointo.github.io/KmapWebsite}}}} \ \ \ \footnotesize \emph{HTML/CSS \& JavaScript}
				}
				{\textbf{Winter 2018}}
		\item
			Implemented the Quine-McCluskey algorithm in JavaScript to compute the \emph{Sum of Products} and \emph{Product of Sums} from a set of truth-table inputs, primarily for use in digital logic design.

\end{itemize}

\end{document}

