\documentclass[10pt,letterpaper]{article}
\usepackage[letterpaper,margin=0.5in]{geometry}
\usepackage[utf8]{inputenc}
\usepackage{mdwlist}
\usepackage[T1]{fontenc}
\usepackage[usenames, dvipsnames]{color}
\usepackage{textcomp}
\usepackage{xcolor}
\usepackage[hidelinks]{hyperref}
\hypersetup{
	colorlinks = true,
	urlcolor={black},
}
\usepackage{tgpagella}
\pagestyle{empty}
\setlength{\tabcolsep}{0em}
\renewcommand{\baselinestretch}{1.025}

\newcommand{\items}[2]
{
	% @ is a column separator 
	\begin{tabular*}{\linewidth}{l @{\extracolsep{\fill}} r}
		#1 & #2 \\
	\end{tabular*}
}


\newcommand{\header}[2]
{
	% @ is a column separator 
	\begin{tabular*}{\linewidth}{l @{\extracolsep{\fill}} r}
		 #1 & #2 \\
	\end{tabular*}
}

\newcommand{\sectionbreak}
{
	\vspace{-1.2em}
	\rule{\textwidth}{1.7pt}
	\vspace{-1.7em}
}

\newcommand{\twocol}[2]
{
	\begin{tabular*}{\linewidth}{l @{\hspace{108.5pt}} l}
		 #1 & #2 \\
	\end{tabular*}
	\vspace{-15pt}

}

\begin{document}

\begin{center}
{\LARGE \textbf{Aashir Farooqi}}

\vspace{0.5em}
\ (949)-226-9612 \textbar 
\ afarooqi@ucdavis.com \textbar
\ \href{https://github.com/AashPointO}{\emph{\underline{\smash{https://github.com/AashPointO}}}}
\\
\end{center}
\vspace{-20pt}


\subsection*{Education}
\sectionbreak

\begin{itemize}

\item[] 
	\header
		{\textbf{University of California, Davis}}
		{\textbf{Fall 2016 - Summer 2020}}
\item[]
	\vspace{-2.5pt}
	\items
		{B.S. Computer Engineering}
		{}
	\items
		{\emph{GPA:} 3.4}
{\vspace{-0.6em}}
	
\end{itemize}

\vspace{-24.65pt}



\subsection*{Technical Skills}
\sectionbreak

\begin{itemize}
	\item[]
		\twocol
		{\textbf{Programming/Markup Languages:}}
		{\hspace{20pt}\hspace{42pt} \textbf{Technological Environments/Libraries:}}
	\item[]
		\begin{tabular*}{\linewidth}{l @{\hspace{152.5pt} \hspace{39pt}} l}
			 C/C++, Rust, Bash, MIPS, RISC-V,   & SPICE, Linux/UNIX, ModelSim, \\
			 Verilog, HTML/CSS, MATLAB, &  Vim, Android Studio, Quartus,  \\
			 Python, Java, JavaScript, \LaTeX. & Git, EAGLE, Altium.
		\end{tabular*}		
\end{itemize}

\vspace{-1.5em}

\subsection*{Experience}
\sectionbreak



\begin{itemize}
	\item[]
		\items
			{\textbf{Embedded Software Engineer - Research Assistant}} 
			{\textbf{April 2018 - June 2020}}
		\items
		{\textbf{Miller Lab} \ (\href{https://millerlab.faculty.ucdavis.edu}{\small \emph{\underline{\smash{millerlab.faculty.ucdavis.edu})}}} }
			{\textbf{Auditory Neuroscience and Speech Recognition Lab}} 
		\item
			Independently brought up, prototyped, and implemented a hybrid hardware and software solution to cross-reference external audio and serial data inputs, with our EEG acquisition system in real-time. Brought latency down from the previous iteration by a factor of 10. 
		\item 
			Wrote a Python to MATLAB wrapper which grabs the gaze angle from our eye-tracker through the Lab Streaming Layer API. Designed as a proof of concept to be incorporated into future studies which will require eye tracking data.
		\item
			Wrote embedded firmware code, created hardware schematics, and designed/assembled multiple PCBs in EAGLE.
\end{itemize}

\hrule

\begin{itemize}
	\item[]
		\items
			{\textbf{Software Engineer - Intern}} 
			{\textbf{June 2018 - August 2018}}
		\items
			{\textbf{General Atomics}}
			{\textbf{EMS - Software and Controls}} 
		\item
			Converted mathematical intensive algorithms of the aircraft landing simulation from MATLAB to C++, bringing the runtime of the simulation down by over a factor of 2. My conversion is now used in research and development of the actual aircraft landing system contracted for the world’s most expensive aircraft carriers.
		\item
			Only intern in department of over 20 to earn "MVP" award for saving "hundreds of hours in simulation time and greatly reducing control system tuning efforts".
		\item
			Wrote a variety of unit-tests using a testing-framework to verify my team's simulations.

\end{itemize}

\vspace{-1.5em}

\subsection*{Projects}
\sectionbreak

\begin{itemize}
	\item[]
		\items
			{\textbf{Operating Systems Course Projects:} \ \ \ \footnotesize  \emph{C/C++ \& Linux Command Line} }
			{\textbf{Spring 2020}}
		\item
			Implemented multiple OS functionality spanning the user and kernel space, such as writing a basic Linux shell in C, preemptive scheduling of different processes, support for multi-threaded safe variables, and writing a Linux driver for a FAT16 file system.
		\item 
			Refined my ability in utilizing the GNU Debugger in debugging multi-threaded processes.
\end{itemize}

\hrule 

\begin{itemize}
	\item[]
		\items 
		{
			\textbf{Senior Design Project: }
			\emph{\smash{Smart Dog Collar}} \ \ \ \footnotesize \emph{C \& Verilog}
		}
			{\textbf{Fall 2019 \& Winter 2020}}
		\item 
			Wrote embedded firmware and HDL code onto Cypress's PSoC. Incorporated a BLE module for wakeup interrupts and data transfer from a mobile application to our device. Wrote software modules to interface with external sensors and hardware peripherals, through I$^{2}$C, I$^{2}$S, SPI, and UART. 
		\item
			Designed multiple iterations of PCBs through Altium by consulting data sheet, which I assembled through use of soldering irons and hot air stations, and eventually debugged through use of logic analyzers, oscilloscopes, and more.

\end{itemize}

\hrule

\begin{itemize}
	\item[]
		\items 
		{
			\textbf{IOS Games (Formerly Published): }
			\href{https://appadvice.com/app/round-bound/1369632746}{\emph{\underline{\smash{Round 'a Bound}}}}, 
			\href{https://appadvice.com/app/tic-tac-emoji/1346934986}{\emph{\underline{\smash{Tic-Tac Emoji}}}} \ \ \ \footnotesize  \emph{Swift}
		}
			{\textbf{Winter 2017 \& Spring 2018}}
		\item 
			Utilized the Spritekit API to detect physics collisions between nodes and to exhibit independently made animations and sounds.
		\item
			Incorporated an online leaderboard via a realtime database through Google's Firebase API, which parses through JSON data. 
\end{itemize}

\end{document}

