\documentclass[10pt,letterpaper]{article}
\usepackage[letterpaper,margin=0.5in]{geometry}
\usepackage[utf8]{inputenc}
\usepackage{mdwlist}
\usepackage[T1]{fontenc}
\usepackage[usenames, dvipsnames]{color}
\usepackage{textcomp}
\usepackage{xcolor}
\usepackage[hidelinks]{hyperref}
\hypersetup{
	colorlinks = true,
	urlcolor={black},
}
\usepackage{tgpagella}
\pagestyle{empty}
\setlength{\tabcolsep}{0em}
\renewcommand{\baselinestretch}{1.025}

\newcommand{\items}[2]
{
	% @ is a column separator 
	\begin{tabular*}{\linewidth}{l @{\extracolsep{\fill}} r}
		#1 & #2 \\
	\end{tabular*}
}


\newcommand{\header}[2]
{
	% @ is a column separator 
	\begin{tabular*}{\linewidth}{l @{\extracolsep{\fill}} r}
		 #1 & #2 \\
	\end{tabular*}
}

\newcommand{\sectionbreak}
{
	\vspace{-1.2em}
	\rule{\textwidth}{1.7pt}
	\vspace{-1.7em}
}

\newcommand{\twocol}[2]
{
	\begin{tabular*}{\linewidth}{l @{\hspace{108.5pt}} l}
		 #1 & #2 \\
	\end{tabular*}
	\vspace{-15pt}

}

\begin{document}

\begin{center}
{\LARGE \textbf{Aashir Farooqi}}

\vspace{0.5em}
\ (949)-226-9612 \textbar 
\ afarooqi@ucdavis.com \textbar
\ \href{https://github.com/AashPointO}{\emph{\underline{\smash{https://github.com/AashPointO}}}}
\\
\end{center}
\vspace{-20pt}


\subsection*{Education}
\sectionbreak

\begin{itemize}

\item[] 
	\header
		{\textbf{University of California, Davis}}
		{\textbf{Davis, CA}}
\item[]
	\vspace{-2.5pt}
	\items
		{B.S. Computer Engineering}
		{Fall 2016 - Summer 2020}
	\items
		{\emph{GPA:} 3.36}
{\vspace{-0.6em}}
	
\end{itemize}

\vspace{-27.65pt}



\subsection*{Technical Skills}
\sectionbreak

\begin{itemize}
	\item[]
		\twocol
		{\textbf{Programming/Markup Languages:}}
		{\hspace{20pt} \textbf{Technological Environments/Libraries:}}
	\item[]
		\begin{tabular*}{\linewidth}{l @{\hspace{152.5pt}} l}
			 C/C++, Rust, Bash, RISC-V,   & SPICE, Linux, ModelSim, \\
			 Verilog, HTML/CSS, MATLAB, &  Vim, Android Studio, Quartus,  \\
			 Java, JavaScript, \LaTeX. & Git, EAGLE, Altium
		\end{tabular*}		
\end{itemize}

\vspace{-1.5em}

\subsection*{Experience}
\sectionbreak



\begin{itemize}
	\item[]
		\items
			{\textbf{Research Assistant}} 
			{\textbf{April 2018 - June 2020}}
		\items
		{\textbf{Miller Lab} \ (\href{https://millerlab.faculty.ucdavis.edu}{\small \emph{\underline{\smash{millerlab.faculty.ucdavis.edu})}}} }
			{\textbf{Auditory Neuroscience and Speech Recognition Lab}} 
		\item
			Independently brought up, prototyped, and implemented a hybrid hardware/firmware solution to cross-reference external audio and serial data inputs, with our EEG acquisition system in real-time. Brought latency down from the previous iteration by a factor of 10. 
		\item 
			Wrote a MATLAB wrapper which grabs the gaze angle from our eye-tracker through the Lab Streaming Layer API. Designed as a proof of concept to be incorporated into future studies which will require eye tracking data.
		\item
			Wrote embedded firmware code, created hardware schematics, and designed/assembled multiple PCBs in EAGLE.
\end{itemize}

\hrule

\begin{itemize}
	\item[]
		\items
			{\textbf{Software Engineering Intern}} 
			{\textbf{June 2018 - August 2018}}
		\items
			{\textbf{General Atomics}}
			{\textbf{EMS - Software and Controls}} 
		\item
			Converted mathematical intensive algorithms of the aircraft landing simulation from MATLAB to C++, bringing the runtime of the simulation down by over a factor of 2. My conversion is now used in research and development of the actual aircraft landing system contracted for the world’s most expensive aircraft carriers.
		\item
			Only intern in department of over 20 to earn "MVP" award for saving "hundreds of hours in simulation time and greatly reducing control system tuning efforts".

\end{itemize}

\vspace{-1.5em}

\subsection*{Projects}
\sectionbreak


\begin{itemize}
	\item[]
		\items 
		{
			\textbf{Senior Design Project: }
			\emph{\smash{Smart Dog Collar}} \ \ \ \footnotesize \emph{C \& Verilog}
		}
			{\textbf{Fall 2019 \& Winter 2020}}
		\item 
			Wrote embedded firmware and HDL code onto Cypress's PSoC. Incorporated a BLE module for wakeup interrupts and data transfer from a mobile application to our device. Communicated with external peripherals such as MEMS mics, accelerometers, and gyrometers through I$^{2}$C, I$^{2}$S, and SPI. 
		\item 
			Designed and assembled multiple iterations of PCBs through Altium, which incorporated the PSoC, external sensors, and a rechargeable battery.

\end{itemize}

\hrule

\begin{itemize}
	\item[]
		\items 
		{
			\textbf{IOS Apps: }
			\href{https://appadvice.com/app/round-bound/1369632746}{\emph{\underline{\smash{Round 'a Bound}}}}, 
			\href{https://appadvice.com/app/tic-tac-emoji/1346934986}{\emph{\underline{\smash{Tic-Tac Emoji}}}} \ \ \ \footnotesize  \emph{Swift}
		}
			{\textbf{Winter 2017 \& Spring 2018}}
		\item 
			Mobile games utilizing the Spritekit API to detect physics collisions between nodes, and to exhibit independently made animations and sounds.
		\item
			Online leaderboard via a realtime database through Google's Firebase API, which parses through JSON data. 
		\item
			Both originally published and reviewed on the App Store, culminating in over 250 downloads.
\end{itemize}

\hrule

\begin{itemize}
	\item[]
		\items
			{	
				\textbf{Website: }
				\href{https://aashpointo.github.io/KmapWebsite/}{\emph{\underline{\smash{aashpointo.github.io/KmapWebsite}}}} \ \ \ \footnotesize \emph{HTML/CSS \& JavaScript}
				}
				{\textbf{Winter 2018}}
		\item
			Given a set of truth table inputs, website outputs the \emph{Sum of Products} and \emph{Product of Sums} equations via the  Quine-McCluskey method. 
		\item 
			Unlike other K-Map Generating websites, mine allows for multiple outputs, an algorithm which is scalable up to an arbitrary number of bits, and a dynamically sizing table through incorporation of JavaScript.

\end{itemize}

\end{document}

