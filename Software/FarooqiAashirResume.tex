\documentclass[10pt,letterpaper]{article}

% ATS Compatibility:
\input{glyphtounicode}
\pdfgentounicode=1

\usepackage[letterpaper,margin=0.5in]{geometry}
\usepackage[utf8]{inputenc}
\usepackage{mdwlist}
\usepackage[T1]{fontenc}
\usepackage[usenames, dvipsnames]{color}
\usepackage{textcomp}
\usepackage{xcolor}
\usepackage{sectsty}
\usepackage[hidelinks]{hyperref}
\hypersetup{
	colorlinks = true,
	urlcolor={black},
}
\usepackage{tgpagella}
\pagestyle{empty}
\setlength{\tabcolsep}{0em}
\renewcommand{\baselinestretch}{1.025}

\subsectionfont{\fontsize{11}{15}\selectfont}


\newcommand{\items}[2]
{
	% @ is a column separator 
	\begin{tabular*}{\linewidth}{l @{\extracolsep{\fill}} r}
		#1 & #2 \\
	\end{tabular*}
}


\newcommand{\header}[2]
{
	% @ is a column separator 
	\begin{tabular*}{\linewidth}{l @{\extracolsep{\fill}} r}
		\hspace{-27pt} #1 & #2 \\
	\end{tabular*}
}

\newcommand{\sectionbreak}
{
	\vspace{-1.2em}
	\rule{\textwidth}{1.7pt}
	\vspace{-1.7em}
}

\newcommand{\twocol}[2]
{
	\begin{tabular*}{\linewidth}{l @{\hspace{108.5pt}} l}
		 #1 & #2 \\
	\end{tabular*}
	\vspace{-15pt}

}

\begin{document}

\begin{center}
{\LARGE \textbf{Aashir Farooqi}}

\vspace{0.5em}
\ (949)-226-9612 \textbar 
\ afarooqi@ucdavis.edu\textbar
\ \href{https://github.com/AashPointO}{\emph{\underline{\smash{https://github.com/AashPointO}}}}
\\
\end{center}
\vspace{-20pt}


\subsection*{Education}
\sectionbreak

\begin{itemize}

\item[] 
	\header
		{\textbf{University of California, Davis}}
		{\textbf{Fall 2016 - Summer 2020}}
\item[]
	\vspace{-2.5pt}
	\items
		{\textbf{Major:} Computer Engineering, B.S}
		{}
	\items
		{\textbf{GPA:} 3.4}
		{}
	\items
		{\textbf{CS Coursework:} Algorithm Design \& Analysis, Applied Linear Algebra, Operating Systems, Networks.}
		{}
	\items
		{\textbf{EE Coursework:} Embedded Systems, Digital Systems, Circuits, Signal Processing.}
		{}
{\vspace{-0.6em}}
	
\end{itemize}

\vspace{-24.65pt}

\subsection*{Experience}
\sectionbreak

\begin{itemize}
	\item[]
		\header
			{\textbf{Software \& Hardware Engineer - Research Assistant}} 
			{\textbf{April 2018 - June 2020}}
		\header
		{\textbf{Miller Lab} \ (\href{https://millerlab.faculty.ucdavis.edu}{\small \emph{\underline{\smash{millerlab.faculty.ucdavis.edu})}}} }
			{\textbf{Auditory Neuroscience \& Speech Recognition Lab}} 
		\item
			Independently brought up, prototyped, and developed a real-time, efficient solution to cross reference external audio inputs with an EEG acquisition system by leveraging low-overhead algorithms in C onto an embedded system in a Linux environment. Brought latency down from the previous iteration by a factor of 10.
		\item 
			Manipulated large amounts of data from conducting EEGs on participants in MATLAB in a Linux environment, and conducted the data conditioning algorithm of PCA to manage these large datasets, by extracting important features in response to auditory stimuli from both datasets we collected, and third-party datasets.
		\item 
			Taught myself networking principles, such as TCP/IP communication, to communicate with an external eye-tracking system through a third-party API, for use in behavioral studies.
\end{itemize}

\hrule

\begin{itemize}
	\item[]
		\header
			{\textbf{Software Engineer - Intern}} 
			{\textbf{June 2018 - August 2018}}
		\header
			{\textbf{General Atomics}}
			{\textbf{EMS - Software and Controls}} 
		\item
			Leveraged object-oriented and algorithm design principles to convert the code base for an aircraft landing from MATLAB to C++, bringing the runtime of the simulation down 
			by a factor of 2. Despite tight time constraints and minimal assistance, I earned the "MVP" award for saving "hundreds of hours in simulation time and greatly reducing control system tuning efforts".




\end{itemize}

\vspace{-1.5em}

\subsection*{Projects}
\sectionbreak


\begin{itemize}
	\item[]
		\header
		{
			\textbf{Algorithm Design \& Embedded Systems}
			\emph{\smash{Relevant Course Projects}} \ \ \ \footnotesize \emph{C++ \& MATLAB}
		}
			{\textbf{Fall 2019 \& Winter 2020}}
		\item 
			Implemented different text search algorithms, such as the Z algorithm, to efficiently find all occurrences of a piece of text in linear time.
		\item
			Implemented parts of Google’s PageRank algorithm, and studied the general usage of web crawling to create the sparse matrices needed in this algorithm.
		\item 
			Implemented the REST API onto an embedded system to interact with my front-end website, and gather real time weather forecasts.

\end{itemize}

\hrule

\begin{itemize}
	\item[]
		\header
		{
			\textbf{Mobile Applications (IOS): }
			\href{https://appadvice.com/app/round-bound/1369632746}{\emph{\underline{\smash{Round 'a Bound}}}}, 
			\href{https://appadvice.com/app/tic-tac-emoji/1346934986}{\emph{\underline{\smash{Tic-Tac Emoji}}}} \ \ \ \footnotesize  \emph{Swift}
		}
			{\textbf{Winter 2017 \& Spring 2018}}
		\item 
			Utilized the Spritekit API to detect physics collisions between nodes and to exhibit independently made animations and sounds, all as a means of being proactive in improving the end-user experience.
		\item
			Incorporated an online leaderboard via a realtime database through Google's third-party API.
		\item
			Apps originally published and reviewed on the App Store, culminating in over 250 downloads.
\end{itemize}

\hrule

\begin{itemize}
	\item[]
		\header
			{	
				\textbf{Website: }
				\href{https://aashpointo.github.io/KmapWebsite/}{\emph{\underline{\smash{aashpointo.github.io/KmapWebsite}}}} \ \ \ \footnotesize \emph{HTML/CSS \& JavaScript}
				}
				{\textbf{Winter 2018}}
		\item
			Implemented the Quine-McCluskey algorithm in JavaScript to compute the \emph{Sum of Products} and \emph{Product of Sums} from a set of truth-table inputs.
		\item
			Made as a response to the clunky end-user experience of similar tools on other websites, and opted to use an algorithm which is scalable up to an arbitrary number of inputs, despite the added complexity of its implementation.

\end{itemize}

\vspace{-1.5em}
\subsection*{Technical Skills}
\sectionbreak

\begin{itemize}

	\item
		\textbf{Proficient:} C/C++, MATLAB, Python, Bash, RISC-V. 
	\item
		\textbf{Familiar:} Java, Rust, Swift, R, \LaTeX.
\end{itemize}

\end{document}

