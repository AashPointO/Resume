\documentclass[10pt,letterpaper]{article}

% ATS Compatibility:
\input{glyphtounicode}
\pdfgentounicode=1

\usepackage[letterpaper,margin=0.5in]{geometry}
\usepackage[utf8]{inputenc}
\usepackage{mdwlist}
\usepackage[T1]{fontenc}
\usepackage[usenames, dvipsnames]{color}
\usepackage{textcomp}
\usepackage{xcolor}
\usepackage{sectsty}
\usepackage[hidelinks]{hyperref}
\hypersetup{
	colorlinks = true,
	urlcolor={black},
}
\usepackage{tgpagella}
\pagestyle{empty}
\setlength{\tabcolsep}{0em}
\renewcommand{\baselinestretch}{1.025}

\subsectionfont{\fontsize{11}{15}\selectfont}


\newcommand{\items}[2]
{
	% @ is a column separator 
	\begin{tabular*}{\linewidth}{l @{\extracolsep{\fill}} r}
		#1 & #2 \\
	\end{tabular*}
}


\newcommand{\header}[2]
{
	% @ is a column separator 
	\begin{tabular*}{\linewidth}{l @{\extracolsep{\fill}} r}
		\hspace{-27pt} #1 & #2 \\
	\end{tabular*}
}

\newcommand{\sectionbreak}
{
	\vspace{-1.2em}
	\rule{\textwidth}{1.7pt}
	\vspace{-1.7em}
}

\newcommand{\twocol}[2]
{
	\begin{tabular*}{\linewidth}{l @{\hspace{108.5pt}} l}
		#1 & #2 \\
	\end{tabular*}
	\vspace{-15pt}

}

\begin{document}

\begin{center}
	{\LARGE \textbf{Aashir Farooqi}}

	\vspace{0.5em}
	\ (949)-226-9612 \textbar 
	\ afarooqi@ucdavis.edu\textbar
	\ \href{https://github.com/AashPointO}{\emph{\underline{\smash{https://github.com/AashPointO}}}}
	\\
\end{center}
\vspace{-20pt}


\subsection*{Education}
\sectionbreak

\begin{itemize}

	\item[] 
		\header
		{\textbf{University of California, Davis}}
		{\textbf{Fall 2016 - Summer 2020}}
	\item[]
		\vspace{-2.5pt}
	\items
		{\textbf{Major:} Computer Engineering, B.S}
		{}
	\items
		{\textbf{GPA:} 3.4}
		{}
	\items
		{\textbf{CS Coursework:} Algorithm Design \& Analysis, Applied Linear Algebra, Operating Systems, Networks.}
		{}
	\items
		{\textbf{EE Coursework:} Embedded Systems, Digital Systems, Circuits, Signal Processing.}
		{}
		{\vspace{-0.6em}}

\end{itemize}

\vspace{-24.65pt}

\subsection*{Experience}
\sectionbreak

\begin{itemize}
	\item[]
		\header
		{\textbf{Firmware \& Hardware Engineer - Research Assistant}} 
		{\textbf{April 2018 - June 2020}}
		\header
		{\textbf{Miller Lab} \ (\href{https://millerlab.faculty.ucdavis.edu}{\small \emph{\underline{\smash{millerlab.faculty.ucdavis.edu})}}} }
		{\textbf{Auditory Neuroscience \& Speech Recognition Lab}} 
	\item
		Independently brought up, prototyped, and delivered a real-time time solution to cross-reference external audio inputs with a Brain-Computer Interface by writing embedded firmware code in C onto an Arduino board, and designing a circuit for a low latency optimized, single-bit ADC. Implemented low-overhead data structures/algorithms, and introduced large amounts of code modularity for maintainability. Brought latency down from the previous iteration by a factor of 10.
	\item 
		Configured the board bringup for a custom embedded Linux RTOS and cross-toolchain using the Yocto Project, including configurations for a bootloader, operating system, and other optimizations based on the architecture of the embedded system. Gained insight on writing Linux kernel modules, IPC, and device drivers.
	\item 
		Taught myself TCP/IP communication in the scripting languages of MATLAB and Python, to implement an external eye-tracking system for use in behavioral studies.
	\item 
		Debugged software issues using tools such as GDB and Valgrind, alongside hardware issues through use of various lab bench equipment (oscilloscopes, function generators, logic analyzers).
\end{itemize}

\hrule

\begin{itemize}
	\item[]
		\header
		{\textbf{Software Engineer - Intern}} 
		{\textbf{June 2018 - August 2018}}
		\header
		{\textbf{General Atomics}}
		{\textbf{EMS - Software and Controls}} 
	\item
		Brought up a new iteration of an aircraft landing simulation in C++, by studying the in-house developed physics models reflecting the interaction of the aircraft and custom motors/actuators, and consulting the original codebase in MATLAB. Brought the runtime of the simulation down by a factor of 2, alongside the inclusion of more accurate physics models and algorithms. Despite tight time constraints and minimal assistance, I earned the "MVP" award for saving "hundreds of hours in simulation time and greatly reducing control system tuning efforts". 

	\item
		Leveraged object-oriented design patterns, implemented large amounts of code modularity, and designed data structures with clean interfaces, for emphasis on maintainability.


\end{itemize}

\vspace{-1.5em}

\subsection*{Projects}
\sectionbreak


\begin{itemize}
	\item[]
		\header
		{
			\textbf{Smart Dog Collar}
			\emph{\smash{Senior Design Project}} \ \ \ \footnotesize \emph{C \& Verilog}
		}
		{\textbf{Fall 2019 \& Winter 2020}}
	\item 
		Wrote embedded firmware code in C, with peripheral interfacing using SPI, I2C, I2S, and UART, to track ambient noise,
		and cycle between different power modes for the SoC and peripherals.
	\item 
		Brought up, assembled, and debugged multiple iterations of hardware to house the microcontroller and various sensors, leveraging circuit analysis techniques, a myriad of datasheets, and various lab bench equipment.

\end{itemize}

\hrule

\begin{itemize}
	\item[]
		\header
		{
			\textbf{Operating Systems \& Embedded Systems}
			\emph{\smash{Relevant Course Project}} \ \ \ \footnotesize \emph{C++}
		}
		{\textbf{Fall 2019 \& Winter 2020}}
	\item 
		Wrote C++ code to implement functionality of a Unix based OS, including the implementation of multithreading, building a device driver for a FAT16 file system, and writing a Unix shell.
	\item 
		Built a smart, IoT AC unit onto a 32-bit MSP430 microcontroller, with a feedback loop to control an IR emitter over SPI, based on data collected from a temperature sensor over I2C, and real-time time weather forecast over the REST API. 

\end{itemize}

\hrule

\begin{itemize}
	\item[]
		\header
		{
			\textbf{Mobile Applications (IOS): }
			\href{https://appadvice.com/app/round-bound/1369632746}{\emph{\underline{\smash{Round 'a Bound}}}}, 
			\href{https://appadvice.com/app/tic-tac-emoji/1346934986}{\emph{\underline{\smash{Tic-Tac Emoji}}}} \ \ \ \footnotesize  \emph{Swift}
		}
		{\textbf{Winter 2017 \& Spring 2018}}
	\item 
		Successfully delivered the products through their full life cycle, by utilizing different high level APIs to detect physics collisions, exhibiting independently made animations and sound, and incorporating an online leaderboard via a real-time time database. While formerly published, the apps culminated in over 250 downloads.
\end{itemize}


\end{document}

