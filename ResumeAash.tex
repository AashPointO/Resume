\documentclass[10pt,letterpaper]{article}
\usepackage[letterpaper,margin=0.5in]{geometry}
\usepackage[utf8]{inputenc}
\usepackage{mdwlist}
\usepackage[T1]{fontenc}
\usepackage[usenames, dvipsnames]{color}
\usepackage{textcomp}
\usepackage{xcolor}
\usepackage[hidelinks]{hyperref}
\hypersetup{
	colorlinks = true,
	urlcolor={black},
}
\usepackage{tgpagella}
\pagestyle{empty}
\setlength{\tabcolsep}{0em}
\renewcommand{\baselinestretch}{1.025}

\newcommand{\items}[2]
{
	% @ is a column separator 
	\begin{tabular*}{\linewidth}{l @{\extracolsep{\fill}} r}
		#1 & #2 \\
	\end{tabular*}
}


\newcommand{\header}[2]
{
	% @ is a column separator 
	\begin{tabular*}{\linewidth}{l @{\extracolsep{\fill}} r}
		 #1 & #2 \\
	\end{tabular*}
}

\newcommand{\sectionbreak}
{
	\vspace{-1.2em}
	\rule{\textwidth}{1.7pt}
	\vspace{-1.7em}
}

\newcommand{\twocol}[2]
{
	\begin{tabular*}{\linewidth}{l @{\hspace{108.5pt}} l}
		 #1 & #2 \\
	\end{tabular*}
	\vspace{-12.5pt}

}

\begin{document}

\begin{center}
{\LARGE \textbf{Aashir Farooqi}}

\vspace{0.5em}
\ (949)-226-9612 \textbar 
\ aashir24@gmail.com \textbar
\ \href{https://github.com/AashPointO}{\emph{\underline{\smash{https://github.com/AashPointO}}}}
\\
\end{center}
\vspace{-20pt}


\subsection*{Education}
\sectionbreak

\begin{itemize}

\item[] 
	\header
		{\textbf{University of California, Davis}}
		{\textbf{Davis, CA}}
\item[]
	\vspace{-2.5pt}
	\items
		{ \emph{College of Engineering:} B.S. Computer Engineering}
		{\emph{GPA:} 3.23}
	\items
		{\emph{Expected graduation:} \ \ \ \ June 2020}
{\vspace{-0.6em}}
	
\end{itemize}

\vspace{-27.65pt}



\subsection*{Technical Skills}
\sectionbreak

\begin{itemize}
	\item[]
		\twocol
		{\textbf{Programming/Markup Languages:}}
		{\hspace{20pt} \textbf{Technological softwares/libraries:}}
	\item[]
		\begin{tabular*}{\linewidth}{l @{\hspace{80pt}} l}
			 \emph{Fluent}: C/C++. & SpriteKit, Unix-Based OS's, XCode, \\
			 \emph{Advanced}: Verilog, HTML/CSS, Bash, MATLAB. &  Vim, Android Studio, Arduino IDE,  \\
			 \emph{Beginner}: Rust, Java, JavaScript \LaTeX. & Chrome Console, GitHub.
		\end{tabular*}		
\end{itemize}

\vspace{-1.5em}

\subsection*{Experience}
\sectionbreak

\begin{itemize}
	\item[]
		\items
			{\textbf{Software Engineering Intern}} 
			{\textbf{June 2018 - August 2018}}
		\items
			{\textbf{General Atomics}}
			{\textbf{EMS - Software and Controls}} 
		\item
			I converted thousands of lines of code of the mathematical intensive algorithms of an aircraft landing simulation from MATLAB to C, enabling the simulation to run more than twice as fast as its original speed. My conversion is now used in research and development of the actual aircraft landing system contracted for the world’s most expensive aircraft carriers.
		\item
			Only intern in department of 20 to earn "Most Valuable Player" award for saving "hundreds of hours in simulation time and greatly reduce control system tuning efforts."

\end{itemize}

\hrule

\begin{itemize}
	\item[]
		\items
			{\textbf{Research Assistant}} 
			{\textbf{April 2018 - Present}}
		\items
		{\textbf{Miller Lab} \ (\href{https://millerlab.faculty.ucdavis.edu}{\small \emph{\underline{\smash{millerlab.faculty.ucdavis.edu})}}} }
			{\textbf{Auditory Neuroscience and Speech Recognition Lab}} 
		\item
			I created a device to take serial data from MATLAB and analog data from multiple audio jacks, designate each input with a corresponding code, and output them in a priority queue to our EEG Acquisition software. Accomplished by programming an ATmega2560, which is fed inputs from my custom designed/assembled PCB centered around an LM339N. This low-jitter, highly-responsive approach allows us to accurately, and reliably, monitor neural responses to external stimuli in our behavioral studies.
		\item 
			I wrote a MATLAB wrapper which grabs the gaze angle from our eye tracker through the Lab Streaming Layer API. Designed as a proof of concept to be incorporated into future studies which will require eye tracking data.
\end{itemize}

\vspace{-1.5em}

\subsection*{Independent Projects (\emph{source code available on \emph{\href{https://github.com/aashpointo}{\emph{\underline{\smash{GitHub}}}}})}}
\sectionbreak

\begin{itemize}
	\item[]
		\items 
		{
			\textbf{IOS Apps: }
			\href{https://itunes.apple.com/us/app/round-bound/id1369632746?mt=8}{\emph{\underline{\smash{Round 'a Bound}}}}, 
			\href{https://itunes.apple.com/us/app/tic-tac-emoji/id1346934986?mt=8}{\emph{\underline{\smash{Tic-Tac Emoji}}}} \ \ \ \footnotesize \emph{Swift}
		}
			{\textbf{Winter 2017 \& Spring 2018}}
		\item 
			Mobile games utilizing the Spritekit API to detect physics collisions between nodes, and to exhibit independently made animations and sounds.
		\item
			Online leaderboard via a realtime database through Google's Firebase API, which parses through JSON data. 
		\item
			Both originally published and reviewed on the App Store, culminating in over 250 downloads.
\end{itemize}

\hrule

\begin{itemize}
	\item[]
		\items
			{	
				\textbf{Websites: }
				\href{https://aashpointo.github.io/KmapWebsite/}{\emph{\underline{\smash{aashpointo.github.io/KmapWebsite}}}} \ \ \ \footnotesize \emph{HTML/CSS \& Javascript}
				}
				{\textbf{Winter 2018}}
		\item
			Given a set of of truth table inputs, my website outputs the \emph{Sum of Products} and \emph{Product of Sums} equations via the  Quine-McCluskey method. 
		\item 
			Unlike other K-Map Generating websites, mine allows for multiple outputs, an algorithm which is scalable up to an arbitrary number of bits, and a dynamically sizing table through my incorporation of Javascript.

\end{itemize}

\end{document}

