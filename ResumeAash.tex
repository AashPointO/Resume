\documentclass[10pt,letterpaper]{article}
\usepackage[letterpaper,margin=0.5in]{geometry}
\usepackage[utf8]{inputenc}
\usepackage{mdwlist}
\usepackage[T1]{fontenc}
\usepackage[usenames, dvipsnames]{color}
\usepackage{textcomp}
\usepackage{xcolor}
\usepackage[hidelinks]{hyperref}
\hypersetup{
	colorlinks = true,
	urlcolor={black},
}
\usepackage{tgpagella}
\pagestyle{empty}
\setlength{\tabcolsep}{0em}

\newcommand{\items}[2]
{
	% @ is a column separator 
	\begin{tabular*}{\linewidth}{l @{\extracolsep{\fill}} r}
		#1 & #2 \\
	\end{tabular*}
}


\newcommand{\header}[2]
{
	% @ is a column separator 
	\begin{tabular*}{\linewidth}{l @{\extracolsep{\fill}} r}
		 #1 & #2 \\
	\end{tabular*}
}

\newcommand{\sectionbreak}
{
	\vspace{-1.2em}
	\rule{\textwidth}{1.7pt}
	\vspace{-1.7em}
}

\newcommand{\twocol}[2]
{
	\begin{tabular*}{\linewidth}{l @{\hspace{108.5pt}} l}
		 #1 & #2 \\
	\end{tabular*}
	\vspace{-12.5pt}

}

\begin{document}

\begin{center}
{\LARGE \textbf{Aashir Farooqi}}

\vspace{0.5em}
39 Needle Grass, Irvine CA 92603 \textbar 
\ (949)-226-9612 \textbar 
\ aashir24@gmail.com \textbar
\ \href{https://github.com/AashPointO}{\emph{\underline{\smash{https://github.com/AashPointO}}}}
\\
\end{center}
\vspace{-20.65pt}

\subsection*{Objective}
\sectionbreak

\begin{itemize}
	\item[] Seeking out an internship or part-time position in a software engineering or computer hardware related field.

\end{itemize}

\vspace{-17.65pt}

\subsection*{Education}
\sectionbreak

\begin{itemize}

\item[] 
	\header
		{\textbf{University of California, Davis}}
		{\textbf{Davis, CA}}
\item[]
	\vspace{-2.5pt}
	\items
		{ \emph{College of Engineering:} B.S. Computer Engineering}
		{\emph{GPA:} 3.00}
	\items
		{\emph{Expected graduation:} \ \ \ \ June 2020}
		{}
\item[]
	\items
		{\emph{Relevant Courses:} Data Objects and Structures (C/C++), Object-Oriented Programming (Rust), \\ \hspace{74pt} Programming/Problem Solving (C), Discrete Mathematics, Digital Systems, Circuits.} 
		{}
	\items
		{\emph{Self-Taught:} \hspace{21pt} Swift/IOS development, vimscript, and \LaTeX.}
		{}
	
\end{itemize}

\vspace{-27.65pt}



\subsection*{Technical Skills}
\sectionbreak

\begin{itemize}
	\item[]
		\twocol
		{\textbf{Programming/Markup Languages:}}
		{\textbf{Technological softwares/libraries:}}
	\item[]
		\begin{tabular*}{\linewidth}{l @{\hspace{75pt}} l}
			 \emph{Fluent}: C/C++. & SpriteKit, Unix, XCode, Vim, Android Studio,\\
			 \emph{Advanced}: Rust, Java, Swift, Bash, MATLAB. &  Arduino IDE, Microsoft Office.  \\
			 \emph{Beginner}: HTML, \LaTeX.
		\end{tabular*}		
\end{itemize}

\vspace{-1.5em}

\subsection*{Experience}
\sectionbreak

\begin{itemize}
	\item[]
		\items
			{\textbf{Software Engineering Intern}} 
			{\textbf{June 2018 - August 2018}}
		\items
			{\textbf{General Atomics}}
			{\textbf{EMS - Software and Controls}} 
		\item
			I converted thousands of lines of code of the mathematical intensive algorithms of an aircraft landing simulation from MATLAB to C, enabling the simulation to run more than twice as fast as its original speed. My conversion is now used in research and development of the actual aircraft landing system contracted for the world’s most expensive aircraft carriers.
		\item
			Only intern in department of 20 to earn "Most Valuable Player" award for saving "hundreds of hours in simulation time and greatly reduce control system tuning efforts."
		\item
			I created and Presented several PowerPoint Presentations detailing the general process of the aircraft landing system, which are now being used for teaching new employees.


\end{itemize}

\hrule

\begin{itemize}
	\item[]
		\items
			{\textbf{Research Assistant}} 
			{\textbf{April 2018 - Present}}
		\items
		{\textbf{Miller Lab} \ (\href{https://millerlab.faculty.ucdavis.edu}{\small \emph{\underline{\smash{millerlab.faculty.ucdavis.edu})}} } }
			{\textbf{Auditory Neuroscience and Speech Recognition Lab}} 
		\item
			Wrote Arduino software which takes in digital inputs from buttons, and analog data from multiple audio jacks, and relays this information to our EEG acquisition program through soldered serial pins. Crucial in accurately interpreting data of brain wave patterns in response to audio cues from our behavioral studies.
		\item 
			Wrote a MATLAB wrapper which grabs the gaze angle from our eye tracker through the Lab Streaming Layer API. Designed as a proof of concept to be incorporated into future studies which will require eye tracking data.
		\item
			Wrote backend database in MATLAB to track participants in our studies.
\end{itemize}

\vspace{-1.5em}

\subsection*{Independent Projects (\emph{code available on \emph{\href{https://github.com/aashpointo}{\emph{\underline{\smash{GitHub}}}}})}}
\sectionbreak

	

\begin{itemize}
	\item[]
		\items 
		{
			\href{https://itunes.apple.com/us/app/round-bound/id1369632746?mt=8}{\emph{\underline{\smash{\textbf{Round 'a Bound}}}}} \ \ \ \footnotesize Swift
		}
			{\textbf{Spring 2018}}
		\item 
			Utilizes Spritekit API in detecting physics collisions between different nodes, and exhibiting custom animations and sounds.
		\item
			Gradients and textures incorporated through self use of Photoshop.
		\item
			Online leaderboard via a realtime database through Google's Firebase API, which parses through JSON data. 
		\item
			Published and reviewed on App Store, with over 60 downloads. 
\end{itemize}

\hrule

\begin{itemize}
	\item[]
		\items
			{
				\href{https://itunes.apple.com/us/app/tic-tac-emoji/id1346934986?mt=8}{\emph{\underline{\smash{\textbf{Tic-Tac Emoji}}}}} \ \ \ \footnotesize Swift
				}
				{\textbf{Winter 2018}}
		\item
			Utilizes SpriteKit API to create aesthetically pleasing objects, fully equipped with custom animations and sounds.	
		\item
			Published and reviewed on App Store, with over 2000 App Store page views. 
\end{itemize}

\end{document}

