\documentclass[10pt,letterpaper]{article}

% ATS Compatibility:
\input{glyphtounicode}
\pdfgentounicode=1

\usepackage[letterpaper,margin=0.5in]{geometry}
\usepackage[utf8]{inputenc}
\usepackage{mdwlist}
\usepackage[T1]{fontenc}
\usepackage[usenames, dvipsnames]{color}
\usepackage{textcomp}
\usepackage{xcolor}
\usepackage{sectsty}
\usepackage[hidelinks]{hyperref}
\hypersetup{
	colorlinks = true,
	urlcolor={black},
}
\usepackage{tgpagella}
\pagestyle{empty}
\setlength{\tabcolsep}{0em}
\renewcommand{\baselinestretch}{1.025}

\subsectionfont{\fontsize{11}{15}\selectfont}


\newcommand{\items}[2]
{
	% @ is a column separator 
	\begin{tabular*}{\linewidth}{l @{\extracolsep{\fill}} r}
		#1 & #2 \\
	\end{tabular*}
}


\newcommand{\header}[2]
{
	% @ is a column separator 
	\begin{tabular*}{\linewidth}{l @{\extracolsep{\fill}} r}
		\hspace{-27pt} #1 & #2 \\
	\end{tabular*}
}

\newcommand{\sectionbreak}
{
	\vspace{-1.2em}
	\rule{\textwidth}{1.7pt}
	\vspace{-1.7em}
}

\newcommand{\twocol}[2]
{
	\begin{tabular*}{\linewidth}{l @{\hspace{108.5pt}} l}
		 #1 & #2 \\
	\end{tabular*}
	\vspace{-15pt}

}

\begin{document}

\begin{center}
{\LARGE \textbf{Aashir Farooqi}}

\vspace{0.5em}
\ (949)-226-9612 \textbar 
\ afarooqi@ucdavis.edu\textbar
\ \href{https://github.com/AashPointO}{\emph{\underline{\smash{https://github.com/AashPointO}}}}
\\
\end{center}
\vspace{-20pt}


\subsection*{Education}
\sectionbreak

\begin{itemize}

\item[] 
	\header
		{\textbf{University of California, Davis}}
		{\textbf{Fall 2016 - Summer 2020}}
\item[]
	\vspace{-2.5pt}
	\items
		{\textbf{Major:} Computer Engineering, B.S}
		{}
	\items
		{\textbf{GPA:} 3.4}
		{}
	\items
		{\textbf{CS Coursework:} Algorithm Design \& Analysis, Operating Systems, Networks.}
		{}
	\items
		{\textbf{EE Coursework:} Embedded Systems, Digital Systems, Circuits, Signal Processing.}
		{}
{\vspace{-0.6em}}
	
\end{itemize}

\vspace{-24.65pt}

\subsection*{Experience}
\sectionbreak

\begin{itemize}
	\item[]
		\header
			{\textbf{Embedded \& Hardware Engineer - Research Assistant}} 
			{\textbf{April 2018 - June 2020}}
		\header
		{\textbf{Miller Lab} \ (\href{https://millerlab.faculty.ucdavis.edu}{\small \emph{\underline{\smash{millerlab.faculty.ucdavis.edu})}}} }
			{\textbf{Auditory Neuroscience \& Speech Recognition Lab}} 
		\item
			Independently brought up, prototyped, and developed a real-time embedded solution to cross-reference external audio inputs with an EEG acquisition system
			by writing embedded firmware code in C and designing/assembling a single-bit ADC circuit. Brought latency down from the previous iteration by a factor of 10. 
			Wrote extensive software documentation and software requirements for code written.
		\item 
			Wrote Bash shell scripts for tests, alongside basic network troubleshooting in a Linux environment.
		\item 
			Built a custom, real-time embedded Linux distributions and cross-toolchains using the Yocto Project, by understanding the underlying hardware requirements of the target embedded system, and host system.
		\item 
			Taught myself networking principles, such as TCP/IP communication, to communicate with an external eye-tracking system for use in behavioral studies. 


\end{itemize}

\hrule

\begin{itemize}
	\item[]
		\header
			{\textbf{Software Engineer - Intern}} 
			{\textbf{June 2018 - August 2018}}
		\header
			{\textbf{General Atomics}}
			{\textbf{EMS - Software and Controls}} 
		\item
			Leveraged object-oriented and algorithm design principles to convert the code base for an aircraft landing from MATLAB to C++, bringing the runtime of the simulation down 
			by a factor of 2. Despite tight time constraints and minimal assistance, I earned the "MVP" award for saving "hundreds of hours in simulation time and greatly reducing control system tuning efforts".




\end{itemize}

\vspace{-1.5em}

\subsection*{Projects}
\sectionbreak


\begin{itemize}
	\item[]
		\header
		{
			\textbf{Smart Dog Collar}
			\emph{\smash{Senior Design Project}} \ \ \ \footnotesize \emph{C \& Verilog}
		}
			{\textbf{Fall 2019 \& Winter 2020}}
		\item 
			Designed multiple finite state machines based on different power modes of the sensors, which were implemented in a combination of Verilog, and embedded C.
		\item 
			Wrote embedded firmware code in C and HDL code in Verilog onto Cypress's Programmable-SoC. j
		\item 
			Implemented a BLE module for wakeup interrupts and data transfer from a mobile application to our device. 
		\item 
			Communicated with external peripherals such as MEMS mics, accelerometers, and gyrometers through I$^{2}$C, I$^{2}$S, SPI, and UART. 
		\item 
			Designed/assembled multiple iterations of PCBs in Altium, and conducted hardware level debugging.

\end{itemize}

\hrule

\begin{itemize}
	\item[]
		\header
		{
			\textbf{Mobile Applications (IOS): }
			\href{https://appadvice.com/app/round-bound/1369632746}{\emph{\underline{\smash{Round 'a Bound}}}}, 
			\href{https://appadvice.com/app/tic-tac-emoji/1346934986}{\emph{\underline{\smash{Tic-Tac Emoji}}}} \ \ \ \footnotesize  \emph{Swift}
		}
			{\textbf{Winter 2017 \& Spring 2018}}
		\item 
			Utilized the Spritekit API to detect physics collisions between nodes and to exhibit independently made animations and sounds.
		\item
			Incorporated an online leaderboard via a realtime database through Google's Firebase API.
		\item
			Apps originally published and reviewed on the App Store, culminating in over 250 downloads.
\end{itemize}

\hrule

\begin{itemize}
	\item[]
		\header
			{	
				\textbf{Website: }
				\href{https://aashpointo.github.io/KmapWebsite/}{\emph{\underline{\smash{aashpointo.github.io/KmapWebsite}}}} \ \ \ \footnotesize \emph{HTML/CSS \& JavaScript}
				}
				{\textbf{Winter 2018}}
		\item
			Implemented the Quine-McCluskey algorithm in JavaScript to compute the \emph{Sum of Products} and \emph{Product of Sums} from a set of truth-table inputs.

\end{itemize}

\vspace{-1.5em}
\subsection*{Technical Skills}
\sectionbreak

\begin{itemize}

	\item
		\textbf{Proficient:} C/C++, Verilog, MATLAB, Bash, RISC-V. 
	\item
		\textbf{Familiar:} Python, Java, Rust, Swift, R, \LaTeX.
\end{itemize}

\end{document}

